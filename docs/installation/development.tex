\section{Development}

The development environment is primarily used when updating the code.  However with it being the only environment which works on the Windows operating system and also not requiring the use of Docker it may be of more use.  This environment should not be made publicly accessible as it is not securely configured.

\subsection{Installation}

The application is built using Python and Node.  As such it is assumed that Python\footnote{https://www.python.org/downloads/} (v3.6+), Pip\footnote{Is included with Python}, Node\footnote{https://nodejs.org/en/download/} and npm\footnote{Is included with Node} are installed and correctly configured.  Download pages for these can be found in the footnotes.

The installation process is entirely performed on the command line.  If you are unfamiliar with the command line there is a version of this program hosted at commute.napier.ac.uk which requires no set up.

The commands used here were tested on a Linux environment.  While Windows alternatives are provided these are untested and may require some modification.

\subsubsection{Command line setup}

To install the application the following environment variables must be set.  These must be set each time the program is run, or if using Linux can be saved in the .bashrc, .bash\_profile or other configuration files.

\begin{minted}{bash}
	--- Linux ---
	export APP_SETTINGS=Development
	export SECRET_KEY="change_me"
	export FLASK_APP=$(pwd)/main.py
	
	--- Windows ---
	setx APP_SETTINGS Development
	setx SECRET_KEY "change_me"
	setx FLASK_APP %cd%/main.py
\end{minted}

After this the python requirements must be installed.  Do to this move into the flask/app directory.

\begin{minted}{bash}
	--- Linux ---
	cd flask/app/
	
	--- Windows ---
	cd flask\app\
\end{minted}

While in this directory run the following command.  This command should be run inside a virtual environment\footnote{https://docs.python.org/3/tutorial/venv.html}.  A virtual environment can be created by running the following commands.  If a virtual environment is not used administrator privileges are required.

\begin{minted}{bash}
	--- Linux ---
	python -m venv env
	. env/bin/activate
	
	--- Windows ---
	python -m venv env
	env\Scripts\activate.bat
\end{minted}

\subsubsection{Static file generation}

To generate the static files we must first install the Node requirements.  While in the flask/app folder run the following command.

\begin{minted}{bash}
	npm i
\end{minted}

When this command has finished the static files can be generated by running the following command.

\begin{minted}{bash}
	npm run prod
\end{minted}

\subsubsection{Python installation}

Once the virtual environment is set up run the following command.

\begin{minted}{bash}
	pip install -r requirements.txt
\end{minted}

Finally run the following command to launch the server.  This binds to port 5000 and so the website can be accessed at 127.0.0.1:5000.

\begin{minted}{bash}
	flask run
\end{minted}

\subsection{Updating}

To update the code to the latest version run the following commands.

\begin{minted}{bash}
	git pull origin master
	npm run prod
	flask run
\end{minted}
