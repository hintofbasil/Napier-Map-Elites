\section{Useful commands}

\subsection{Manually adding a solutions zip file}

To manually add a solutions file to the server you simply need to copy it to \mintinline{bash}{$SOLUTIONS_FOLDER}.  An example is shown below where it is assumed \mintinline{bash}{$SOLUTIONS_FOLDER} has been set to the same value as the remote server and \mintinline{bash}{$SOLUTIONS_FILE}, \mintinline{bash}{$CSV_FILE} and \mintinline{bash}{$REMOTE_USERNAME} have been set.

\begin{minted}{bash}
mv "$SOLUTIONS_FILE" "$(md5sum $CSV_FILE | cut -d ' ' -f 1).zip"
scp "$SOLUTIONS_FILES" "$REMOTE_USERNAME@commute.napier.ac.uk:$SOLUTIONS_FOLDER"
\end{minted}

\subsection{Manually clearing locks}

Lock files are created when solution files are uploaded.  These locks are deleted on a successful upload but may not be deleted on a failed upload.  They are automatically cleared 15 minutes after they are created but occasionally it is useful to clear the locks manually.  The following command does this.

This command assumes the environment variables have been correctly set.

\begin{minted}{bash}
	docker exec -it napiermapelites_flask_1 /app/clear_locks.py
\end{minted}

\subsection{Clearing server data}

While the server automatically deletes all uploaded solutions at midnight GMT it can be useful to manually clear the server.  The following command achieves this.

This command assumes the environment variables have been correctly set.

\begin{minted}{bash}
	docker exec -it napiermapelites_flask_1 /app/clear_locks.py --ignore-locks --delete-solutions
\end{minted}

\subsection{Migrating data}

To migrate data from one server to another install the application as normal.  Then copy the \$SOLUTIONS\_FOLDER to the new server and set the \$SOLUTIONS\_FOLDER variable on the new server.
