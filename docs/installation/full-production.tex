\section{Full Production}

\textbf{Both production environments are build for Unix systems only.  This means they will not work on Windows.}

\subsection{Certificate Generation}

Generate the SSL certificates.  The generated certificates only last 90 days and must be renewed before that (see \ref{section:certificate-renewal}).  Email reminders will be sent to the registered email address.

\begin{minted}{bash}
docker run --rm -it -p 80:80 -p 443:443 -v /etc/letsencrypt:/etc/letsencrypt certbot/certbot certonly
*** Follow prompts selecting 'spin up a tempory web server' when asked ***
cp /etc/letsencrypt/live/DOMAIN_NAME/* $SSL_LOCATION
\end{minted}

\subsection{Systemd}

Using systemd is the preferred method for launching the project as it allows for simple implementation of environment variables and allows for simple restarts.

\subsubsection{Installation}

First all the files in the systemd directory must be renamed to replace 'APP\_NAME' with the name of your application.

\begin{minted}{bash}
export APP_NAME=YOUR_APP_NAME
cd systemd
rename APP_NAME $APP_NAME $(ls)
cd ..
\end{minted}

The environment variables should then be set in \mintinline{bash}{systemd/$APP_NAME.production.service.d/local.conf}.  Any paths must be absolute paths.

These files must then be copied over to the systemd folder.

\begin{minted}{bash}
cp -r systemd/* /etc/systemd/system/
\end{minted}

Finally systemd needs to be reloaded.

\begin{minted}{bash}
systemctl daemon-reload
\end{minted}

\subsubsection{Launching}

To install the simple production environment the following commands should be run in the root folder of the repository.  These commands require root or being a member of the 'docker' group.

\begin{minted}{bash}
systemctl start $APP_NAME.production.service
\end{minted}

\subsubsection{Updating}

To update the test environment run the following commands while in the repository.  Some of these commands require root.

\begin{minted}{bash}
git pull origin master
docker-compose build
systemctl restart $APP_NAME.production.service
\end{minted}

\subsection{Renewing certificates} \label{section:certificate-renewal}

\begin{minted}{bash}
systemctl stop $APP_NAME.production.service
docker run --rm -it -p 80:80 -p 443:443 -v /etc/letsencrypt:/etc/letsencrypt certbot/certbot renew
cp /etc/letsencrypt/live/DOMAIN_NAME/* $SSL_LOCATION
systemctl start $APP_NAME.production.service
\end{minted}

\subsection{Manual}

Manual installation should typically only be used if systemd is not installed on the system or the above method failed for some reason.

\subsubsection{Installation}

Set the following environment variables.  You may want to put these into your .bashrc file so you don't have to set them each time.  Both paths should be absolute paths.

\begin{minted}{bash}
	export SSL_LOCATION="Location to host ssl certificates."
	export SOLUTIONS_FOLDER="Location to save solutions zip files to."
\end{minted}

\subsubsection{Launching}

\begin{minted}{bash}
	docker-compose -f docker-compose.production.yml build
	docker-compose -f docker-compose.production.yml up -d
\end{minted}

\subsubsection{Updating the production environment}

To update the production environment run the following commands while in the root directory of the repository.  Some of these commands require root.

\begin{minted}{bash}
	git pull origin master
	docker-compose -f docker-compose.production.yml build
	docker-compose down
	docker-compose -f docker-compose.production.yml up -d
\end{minted}

\subsection{Renewing certificates} \label{section:certificate-renewal}

\begin{minted}{bash}
	docker-compose down
	docker run --rm -it -p 80:80 -p 443:443 -v /etc/letsencrypt:/etc/letsencrypt certbot/certbot renew
	cp /etc/letsencrypt/live/DOMAIN_NAME/* $SSL_LOCATION
	docker-compose -f docker-compose.production.yml up -d
\end{minted}
