\section{Simple Production}

\textbf{Both production environments are build for Unix systems only.  This means they will not work on Windows.}

\subsection{Systemd}

Using systemd is the preferred method for launching the project as it allows for simple implementation of environment variables and allows for simple restarts.

\subsubsection{Installation}

First all the files in the systemd directory must be renamed to replace 'APP\_NAME' with the name of your application.

\begin{minted}{bash}
export APP_NAME=YOUR_APP_NAME
cd systemd
rename APP_NAME $APP_NAME $(ls)
cd ..
\end{minted}

The environment variables should then be set in \mintinline{bash}{systemd/$APP_NAME.service.d/local.conf}.  Any paths must be absolute paths.

These files must then be copied over to the systemd folder.

\begin{minted}{bash}
cp -r systemd/* /etc/systemd/system/
\end{minted}

Finally systemd needs to be reloaded.

\begin{minted}{bash}
systemctl daemon-reload
\end{minted}

\subsubsection{Launching}

To install the simple production environment the following commands should be run in the root folder of the repository.  These commands require root or being a member of the 'docker' group.

\begin{minted}{bash}
systemctl start $APP_NAME.service
\end{minted}

\subsubsection{Updating}

To update the test environment run the following commands while in the repository.  Some of these commands require root.

\begin{minted}{bash}
git pull origin master
docker-compose build
systemctl restart $APP_NAME.service
\end{minted}

\subsection{Manual}

Manual installation should typically only be used if systemd is not installed on the system or the above method failed for some reason.

\subsubsection{Launching}

To install the simple production environment the following commands should be run in the root folder of the repository.  These commands require root or being a member of the 'docker' group.

\begin{minted}{bash}
	docker-compose build
	docker-compose up -d
\end{minted}

\subsubsection{Updating}

To update the test environment run the following commands while in the repository.  Some of these commands require root.

\begin{minted}{bash}
	git pull origin master
	docker-compose build
	docker-compose down
	docker-compose up -d
\end{minted}
