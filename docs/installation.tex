\section{Installation}

\subsection{Pre-requisites}

This project is designed to be hosted using Docker\footnote{https://www.docker.com/} and docker-compose\footnote{https://docs.docker.com/compose/}.  While the project is based on Flask and can be hosted using a standard web setup (e.g Nginx with uWSGI) this is not recommended.

Docker can be installed easily using the script at https://get.docker.com/ (although from a security point this is not recommended) or from your package manager.  Windows users should be able to find an installer on the docker website.  Docker-compose is a python package available through pip\footnote{https://packaging.python.org/tutorials/installing-packages/} or your package manager.

\subsection{Downloading files}

All the files required to host the website are available on github at https://github.com/hintofbasil/Napier-Map-Elites.  It is unlikely that pull requests will be accepted into this repository and so you should forking would be a good idea.

\subsection{Different environments}

There are three environments built into the application.  Testing, Development and Production.  The core settings of each environment can be seen in flask/app/config.py

\subsubsection{Development}

This is the environment designed for developing new features.  This does not require Docker and runs locally on a laptop only.  It enables watch features to update the application code when changes are made.

Further details of how to set this up can be found in flask/app/README.md.

This will likely be updated in the future through Flask Docker Boilerplate (see \ref{sec:flask-docker-boilerplate}) to use Docker to help mitigate NPM issues (e.g https://blog.npmjs.org/post/175824896885/incident-report-npm-inc-operations-incident-of).

\subsubsection{Testing}

This is the environment designed for simple testing.  It is designed to be very simple to launch, requiring no configuration, and to be secure enough to expose to the internet.  This environment is hosted using Docker.  It does not enable persistence of uploaded solution files.  It does not require or allow secure connections (https).

\subsubsection{Production}

This is the recommended environment for hosting the website on an internet facing web server.  It enables solution persistence (although it is wiped at 00:00 daily).  It also enables the use of secure connections (https).  This does require some configuration, detailed below, and is only recommended for advanced users.

\subsection{Installing the testing environment}

To install the testing environment the following commands should be run in the root folder of the repository.  These commands require root or being a member of the 'docker' group however the docker group is equivalent to root access with no password requirement and should not be used.

\begin{minted}{bash}
	docker-compose build
	docker-compose up -d
\end{minted}

\subsection{Updating the test environment}

To update the test environment run the following commands while in the repository.  Some of these commands require root.  N.B only one environment can be running at a time.

\begin{minted}{bash}
	git pull origin master
	docker-compose build
	docker-compose down
	docker-compose up -d
\end{minted}

\subsection{Installing the production environment}

Set the following environment variables.  You may want to put these into your .bashrc file so you don't have to set them each time.

\begin{minted}{bash}
	export SSL_LOCATION="Location to host ssl certificates.  Should be an empty folder."
	export SOLUTIONS_FOLDER="Location to save solutions zip files to."
\end{minted}

Create the folders.

\begin{minted}{bash}
	mkdir -p $SSL_LOCATION $SOLUTIONS_FOLDER
\end{minted}

Launch the server

\begin{minted}{bash}
	docker-compose -f docker-compose.production.yml build
	docker-compose -f docker-compose.production.yml up -d
\end{minted}

\subsection{Updating the production environment}

To update the production environment run the following commands while in the repository.  Some of these commands require root.  N.B only one environment can be running at a time.

\begin{minted}{bash}
	git pull origin master
	docker-compose -f docker-compose.production.yml build
	docker-compose down
	docker-compose -f docker-compose.production.yml up -d
\end{minted}
